% \documentclass[german,master,buw]{webisthesis} % Weimar
\documentclass[english,bachelor,fsu]{webisthesis} % Jena
% \documentclass[german,bachelor,ul]{webisthesis} % Leipzig
% \documentclass[german,master,buw,web]{webisthesis} % Weimar, for web page
% \documentclass[german,bachelor,fsu,web]{webisthesis} % Jena, for web page
% \documentclass[german,bachelor,ul,web]{webisthesis} % Leipzig, for web page
%
% Non-default programme
% ---------------------
% \documentclass[english,master,buw]{webisthesis}\global\thesisprogramme{Human-Computer Interaction}
% \documentclass[english,master,buw]{webisthesis}\global\thesisfrontpagefaculty{Faculty of Civil Engineering/Faculty of Media}\global\thesisprogramme{Digital Engineering}
% \documentclass[german,bachelor,buw]{webisthesis}\global\thesisprogramme{Informatik\\Schwerpunkt Medieninformatik}
% \documentclass[german,bachelor,buw]{webisthesis}\global\thesisprogramme{Informatik\\Schwerpunkt Security and Data Science}
%
% When you change the language, pdflatex may halt on recompilation.
% Just hit enter to continue and recompile again. This should fix it.


%
% Values
% ------
\ThesisSetTitle{Analyzing Toxicity on Mastodon}
\ThesisSetKeywords{Toxicity, Mastodon, Instances, Pipeline, Moderation} % only for PDF meta attributes
\ThesisSetLocation{Jena} 

\ThesisSetAuthor{Julian Klüber}
\ThesisSetStudentNumber{201071}
\ThesisSetDateOfBirth{8}{7}{2001}
\ThesisSetPlaceOfBirth{Hünfeld}

% Supervisors should usually be Professors from the candidate's university. A second supervisor is not always needed. 
\ThesisSetSupervisors{Prof.\ Dr.\ Matthias Hagen, Matti Wiegmann{,}\ M.Sc.}

\ThesisSetSubmissionDate{16}{04}{2025}

\usepackage{graphicx}

%
% Suggested Packages
% ------------------
\usepackage[style=alphabetic,natbib=true,backend=biber]{biblatex}
\addbibresource{literature.bib}
%   Allows citing in different ways (e.g., only the authors if you use the
%   citation again within a short time).
%
\usepackage{booktabs}
%    For tables ``looking the right way''.
%
\usepackage{tabularx}
%    Enables tables with columns that automatically fill the page width.
%
% \usepackage[ruled,algochapter]{algorithm2e}
%    A package for pseudo code algorithms.
%
\usepackage{amsmath}
%    For tabular-style formatting of mathematical environments.
%

\usepackage{fontawesome}
%    For lots of awesome glyphs: https://mirror.physik.tu-berlin.de/pub/CTAN/fonts/fontawesome/doc/fontawesome.pdf

%
% Commenting (by your supervisor)
% -------------------------------
\usepackage{xcolor}
\usepackage{soul}
\newcommand{\bscom}[2]{%
  % #1 Original text.
  % #2 Replacement text.
    \st{\scriptsize\,#1}{\color{blue}\scriptsize\,#2}%
  }

% Create links in the pdf document
% Hyperref has some incompatibilities with other packages
% Some other packages must be loaded before, some after hyperref
% Additional options to the hyperref package can be provided in the braces [], like in
% \usehyperref[backref] % This will add back references in the bibliography that some people like ... some don't ... so better ask your supervisor ;-)
\usehyperref

\begin{document}
\begin{frontmatter}
\begin{abstract}
  Social media platforms face significant challenges in mitigating toxic behavior, such as hate speech and harassment, due to their scale and anonymity. While centralized platforms like Twitter/X have been extensively studied, decentralized alternatives like Mastodon remain under-researched, particularly regarding large-scale toxicity patterns and the effectiveness of decentralized moderation. Existing studies on Mastodon are limited by small datasets or short time periods, failing to capture platform-wide trends or the impact of moderation practices across its federated network.

  This study addresses this gap by conducting the first large-scale toxicity analysis on Mastodon, analyzing approximately 18 million posts ("toots") from 1,000 instances throughout 2024. Using a transformer-based toxicity classifier, we examined how toxicity fluctuates during major events and how moderation policies influence these patterns. Our results reveal two key findings: First, toxicity levels spike during politically charged events, such as the 2024 U.S. election. Second, instances with active moderation practices, such as adherence to the Mastodon Covenant or isolation from blocklisted instances, seem to lead to lower toxicity levels. These findings demonstrate that decentralized moderation can effectively reduce toxicity, though inconsistencies across instances remain a challenge. This work provides foundational insights into toxicity dynamics in decentralized social networks and highlights the role of moderation in shaping healthier online communities.
\end{abstract}

\tableofcontents

% \chapter*{Acknowledgements} % optional
% I thank the authors of the webisthesis template for their excellent work!

% \listoffigures % optional, usually not needed

% \listoftables % optional, usually not needed

% \listofalgorithms % optional, usually not needed
%    requires package algorithm2e

% optional: list of symbols/notation (e.g., using the nomencl package) but usually not needed
\end{frontmatter}

\chapter{Introduction}\label{introduction}

Social media has become an essential part of modern life, enabling billions of people to share their stories, connect over shared interests, and participate in discussions on global events. This has resulted in the formation of large online communities. Unlike physical communities, online communities allow for easier and more immediate grouping of individuals, as joining a community often requires minimal effort; simply clicking a button or following a page \cite{ellison:2007}. This ease of access fosters the creation of diverse and dynamic social ecosystems, where users interact under shared norms, behaviors, and communication styles unique to each community.

However, the aggregation of large numbers of individuals in online communities also increases the likelihood of negative behaviors, such as toxicity. Toxicity in online spaces refers to harmful actions, including hate speech, racism, sexism, and other forms of discrimination. These behaviors are often exacerbated by the anonymity and reduced social inhibitions that characterize digital environments \cite{suler:2004}.

\begin{figure}[ht]
    \centering
    \fbox{%
      \parbox{0.8\textwidth}{%
        \centering
        "And two *racist* will light the fire. Of fucking course." \\
        "Fuck all you obedient slaves." \\
        "Rio had maybe five percent of the *racist* and *homophobic* from this disaster."
      }
    }
    \caption{Three toxic examples posted during the 2024 Paris Olympics opening ceremony on Mastodon. Severe harmful words got replaced with the kind of offence.}
\end{figure}

Such behaviors can disrupt community solidarity and harm individual users, making toxicity a significant challenge for social media platforms. To mitigate these issues, online communities establish rules and moderation systems to enforce acceptable behavior. Violations of these rules can result in penalties, such as bans or restrictions, depending on the platform's moderation policies \cite{nicholson:2023}.

To understand the dynamics of online communities and the challenges they face, this case study focuses on Mastodon, a decentralized alternative to traditional social media platforms like Twitter. The recent acquisition of Twitter by Elon Musk has highlighted the risks of centralized social media, where a single individual or entity can exert significant control over platform governance, content moderation, and user experience \cite{zia:2023}. Such centralization can lead to abrupt policy changes, increased misinformation, and heightened toxicity, prompting users to seek alternatives. Mastodon, as a decentralized and federated platform, offers a contrasting model where power is distributed across independently operated instances. However, this decentralization also introduces unique challenges, such as inconsistent moderation standards and the potential for fragmentation within the fediverse \cite{zia:2023}. By examining Mastodon, this study aims to explore how decentralized platforms address toxicity and community management while navigating the complexities of a federated ecosystem.

Like every social media platform, Mastodon offers the possibility to interact with other people by publishing posts, reacting to posts, or sharing posts. However, Mastodon is a federated social media platform. Federation refers to a special kind of decentralization. Traditional social media platforms, such as Twitter, Facebook, or Instagram, have a single central service that all users access. In contrast, Mastodon has multiple services, called instances, which are used by any number of people. These instances can communicate with each other and create a federated network. Users can freely choose an instance based on language, community rules, moderation policies, and topics of interest. Each instance is managed by its own administrators, who set and enforce local rules \cite{mastodon:docs}.

\begin{figure}[tb]
    \centering
    \includegraphics[width=\textwidth]{../material/network_models.jpg}
    \caption{From left to right: Centralized networks connect all through a single controlling hub; Federated networks organize nodes into semi-autonomous interconnected clusters; Distributed networks connect all nodes with multiple pathways. \cite{mastodon:docs}}
    \label{fig:network-models}
\end{figure}

\paragraph{Research Gap and Contribution}
Prior work has established foundational knowledge about Mastodon's structure \cite{zulli:2020,la_cava:2021}, moderation practices \cite{bono:2024,nicholson:2023}, and toxicity analysis methods \cite{fan:2022}. However, no study has systematically examined toxicity patterns across the entire Mastodon network while accounting for its federated architecture. Our research addresses this gap by conducting the first large-scale toxicity analysis of Mastodon. Our research addresses this gap by conducting the first large-scale toxicity analysis of Mastodon. We analyzed a 1\% subsample of 1.8~billion Mastodon posts (called ``toots'') across 1,000~instances collected throughout 2024. The results reveal two key findings: First, toxicity levels show significant spikes during major political and global events, particularly around the U.S.~election period. Second, active moderation practices correlate with lower mean toxicity levels on instances, demonstrating the effectiveness of decentralized moderation approaches.

\enlargethispage{\baselineskip}
\chapter{Related Work} \label{related-work}
\chapter{Defining the Research Scope: Questions, Challenges, and Data}

\section{Problem Statement and Research Question}
Social media platforms face significant challenges in managing toxic behavior, such as hate speech, harassment, and discrimination. While centralized platforms like Twitter have been extensively studied, decentralized alternatives like Mastodon present unique dynamics due to their federated architecture. Mastodon's distributed moderation system, where each instance operates independently, raises questions about how toxicity manifests and evolves across diverse communities.

This study aims to understand the evolution of toxicity on Mastodon over the entire year of~2024. Specifically, we address the following research question:

\begin{quote}
\textbf{How does toxicity vary across Mastodon instances, and how do instance-specific rules and moderation practices influence these patterns?}
\end{quote}

To answer this question, we conduct two experiments:

\begin{enumerate}
    \item \textbf{Toxicity Classifier Selection}: We evaluate three transformer-based toxicity prediction models to identify the most suitable one for our large-scale analysis.
    
    \item \textbf{Toxicity Detection and Analysis}: Using the selected model, we analyze a~1\% subsample of our dataset to detect toxic content and examine its distribution across instances. We further investigate how toxicity levels correlate with instance-specific rules and moderation practices.
\end{enumerate}

\section{Mastodon Dataset} \label{mastodon-dataset}
The dataset used in this study consists of public toots collected from a federated network of instances. The data was gathered using a distributed crawler developed by \citet{ernst:2024}, which collects toots from Mastodon instances' public timelines via their REST API while respecting user privacy settings. The crawler stores the collected data in an Elasticsearch cluster with careful attention to ethical considerations, never publishing raw user data.

\begin{figure}[tb]
    \centering
    \includegraphics[width=\textwidth]{../material/activity_2024.png}
    \caption{Total number of toots in our subset per day.}
    \label{toot-distribution}
    \includegraphics[width=\textwidth]{../material/sensitive_toots.png}
    \caption{Number of Toots marked as sensitive in our subset per day.}
    \label{sensitive-toots}
\end{figure}

As reported by \citet{ernst:2024} in November~2024, the corpus contained 3.6~billion toots collected over 301~days from 1,081~instances. The complete dataset includes detailed metadata about each toot and relationships between instances, with key fields described in Table~\ref{dataset-fields}. After deduplication, this resulted in 174~million unique toots, indicating a duplication rate of approximately 95\%, which reflects the federated nature of Mastodon where content is shared across instances. While the crawler has continued running since that report, our analysis focuses specifically on English-language toots from the complete year~2024, collected from 1,000~fully crawled instances.

\begin{table}[tb]
    \centering\small
    \renewcommand{\arraystretch}{1.3}
    \begin{tabularx}{\textwidth}{lX}
        \toprule
        \textbf{Field} & \textbf{Description} \\
        \midrule
        \texttt{id} & Unique identifier for each toot \\
        \texttt{content} & The toot content in HTML format (converted to plain text for analysis) \\
        \texttt{crawled\_from\_instance} & The instance where the toot was observed \\
        \texttt{instance} & The home instance of the posting user \\
        \texttt{is\_local} & Boolean indicating whether the toot originated on the crawled instance \\
        \texttt{created\_at} & Timestamp of toot creation \\
        \texttt{sensitive} & Flag marking potentially sensitive content \\
        \texttt{spoiler\_text} & Content warnings or spoiler alerts \\
        \bottomrule
    \end{tabularx}
    \caption{Key fields available in the Mastodon dataset with their descriptions.}
    \label{dataset-fields}
\end{table}

Approximately 18\% of toots contain media attachments (mostly images). Because our toxicity detection models just analyze text content, we removed those with media attachments. As well, we filtered out reblogs to reduce the dataset and focus on original toots. The final dataset contains around 1.8~billion toots

\begin{figure}[tb]
    \centering
    \includegraphics[width=\textwidth]{../material/instance_distribution.png}
    \caption{Distribution of Toot Counts per Instance: Comparison of Sources vs. Targets. The left chart shows the number of instances by toot volume for the target instances, while the right chart (log scale) shows the same for the source instances.}
    \label{instance-distribution}
\end{figure}

For our analysis, we use a~1\% subsample of the dataset, resulting in 17,691,031 toots. The subset is created by selecting 10 random toots from each batch of 1000. The batches cover a short time span, ensuring the timeline remains continuous. After subsampling, the subset exhibits the following key characteristics:

\begin{itemize}
\item 5.69\% of toots are flagged as sensitive content.
\item Approximately 20\% originate from users whose home instance is \textit{mastodon.social}.
\item While the origin of the toots is heavily centered on one instance (\textit{mastodon.social}), the distribution of the instances where the toots were posted is more balanced (see Figure~\ref{instance-distribution}).
\item During subsampling, 71 weakly represented instances were removed, leaving 929 instances.
\item The proportion of duplicates is reduced to 50\% (from 95\% in the original dataset).
\end{itemize}

As shown in Figure~\ref{toot-distribution}, the subsample maintains consistent temporal coverage throughout 2024. October and November saw an increase in the number of toots posted. This can be explained by the election campaign and the elections in the USA. There is also an extreme peak in toots marked as sensitive on the 6th of November 2024 (Figure~\ref{sensitive-toots}). As this was the first day after the election in the USA, Donald Trump's victory seems to be the reason. As \citet{zia:2023} already described, \textit{mastodon.social} dominates as users' home instance due to its size (Figure~\ref{instance-distribution}).
\chapter{Choosing Transformer Model} \label{choosing-transformer-model}

To predict the toxicity of the post content, I evaluated three transformer-based models: Detoxify (Original), Detoxify (Unbiased) from Unitary, and Perspective API. These models predict the probability of six to seven toxicity categories.

\section{Annotation} \label{annotation}

All models were evaluated on a specific subset of the dataset. The selected timeframe was the evening of the Olympic Games' opening ceremony, chosen due to the expectation of heightened online discussions and, consequently, an increased presence of toxic content. The dataset covers the period from 20:00 to 23:00 on July 26, 2024, comprising 1,179,897 posts. This context switch which may impact the model performance, but as well shows the model's robustness in a new context.

After the models completed their predictions, a sample was drawn for each model, selecting 25 posts per toxicity category with a predicted probability greater than 0.5 for that category. The selected posts were then concatenated, and duplicates were removed, resulting in a final annotation dataset of 253 posts.

The annotation process was conducted by two researchers using Label Studio. The posts were labeled according to the following categories in the table below. The annotation of hate speech remains a highly subjective task, influenced by individual annotator biases, further affecting consistency and reliability. To ensure the reliability of the annotations, inter-annotator agreement was measured using Cohen's Kappa. The results showed perfect agreement (Cohen's Kappa = 1.0) for the categories of toxic, severe toxic, threat, insult, and identity attack. Near-perfect agreement was achieved for obscene (Cohen's Kappa = 0.9907) and sexually explicit (Cohen's Kappa = 0.9735), indicating a high level of consistency between the annotators.

\begin{table}[h]
    \centering
    \renewcommand{\arraystretch}{1.3}
    \begin{tabularx}{\textwidth}{lX}
        \toprule
        \textbf{Category} & \textbf{Description} \\
        \midrule
        TOXIC & A rude, disrespectful, or unreasonable comment that is likely to make someone leave a discussion. \\
        SEVERE TOXIC & A very hateful, aggressive, or disrespectful comment that is highly likely to push someone away. \\
        IDENTITY ATTACK & Negative or hateful comments targeting someone because of their identity. \\
        INSULT & Insulting, inflammatory, or negative comments towards a person or group. \\
        OBSCENE & Swear words, curse words, or other obscene or profane language. \\
        THREAT & Describes an intention to inflict pain, injury, or violence against an individual or group. \\
        SEXUALLY EXPLICIT & Genital nudity or descriptions of simulated or actual sexual acts. \\
        \bottomrule
    \end{tabularx}
    \caption{Toxicity Categories and Their Descriptions}
\end{table}

\section{Evaluation} \label{evaluation}

In the evaluation of the toxicity detection models, the analysis focused on the probability distributions of toxicity categories across three models: the Perspective API and two Detoxify models (original and unbiased). The evaluation did not involve optimizing prediction thresholds; instead, a fixed threshold of 0.5 was used to compare the F1 scores. The results indicated that the Perspective API model generally outperformed the Detoxify models in terms of F1 score across most toxicity categories. However, a closer examination of the probability distributions revealed notable differences in the models' behavior. The Perspective API model exhibited a widely spread probability distribution, suggesting a more cautious and less confident approach to predictions. In contrast, the Detoxify models demonstrated a more concentrated probability distribution, indicating higher confidence in their predictions.

\begin{figure}[h]
    \centering
    \includegraphics[width=\textwidth]{../material/probability_distribution.png}
    \caption{Probability Distribution of Toxicity category across the three models}
    \label{fig:probability-distribution}
\end{figure}

While the Perspective API model's performance metrics were stronger, its limited accessibility, restricted by a low request rate and lack of open access, posed significant practical limitations for large-scale analysis. On the other hand, the Detoxify models, being open-source, offered unrestricted usage, making them more suitable for extensive studies. Given that the primary objective of this research is not to predict toxicity in individual posts but to analyze broader trends in toxicity across Mastodon instances, the Detoxify models were considered more appropriate. A confident model, such as Detoxify, is better suited for identifying and tracking trends over time.

Between the two Detoxify models, the unbiased version was selected for further analysis due to its superior performance and the inclusion of an additional category, "sexual explicit," which is absent in the original Detoxify model. However, it is important to note that the "severe toxic" label in the Detoxify unbiased model demonstrated poor performance, likely due to the limited number of supporting posts (only 10 in the dataset). Consequently, findings related to this category should be interpreted with caution. Despite this limitation, the Detoxify unbiased model was chosen as the most suitable tool for analyzing toxicity trends in Mastodon communities, balancing performance, accessibility, and practical applicability.


\begin{table}[h!]
\centering
\begin{tabular}{|l|l|c|c|c|c|}
\hline
\textbf{Model} & \textbf{Category} & \textbf{F1} & \textbf{Precision} & \textbf{Recall} & \textbf{Support} \\
\hline
Detoxify Original & & 0.622 & 0.482 & 0.878 & 90 \\
Detoxify Unbiased & toxic & 0.635 & 0.473 & 0.967 & 90 \\
Perspective API & & \textbf{0.664} & 0.526 & 0.900 & 90 \\
\hline
Detoxify Original & & 0.148 & 0.118 & 0.200 & 10 \\
Detoxify Unbiased & severe toxic & 0.000 & 0.000 & 0.000 & 10 \\
Perspective API & & \textbf{0.222} & 0.176 & 0.300 & 10 \\
\hline
Detoxify Original & & \textbf{0.741} & 0.613 & 0.936 & 78 \\
Detoxify Unbiased & obscene & 0.739 & 0.642 & 0.872 & 78 \\
Perspective API & & 0.738 & 0.615 & 0.923 & 78 \\
\hline
Detoxify Original & & 0.462 & 0.429 & 0.500 & 18 \\
Detoxify Unbiased & threat & 0.520 & 0.406 & 0.722 & 18 \\
Perspective API & & \textbf{0.596} & 0.483 & 0.778 & 18 \\
\hline
Detoxify Original & & 0.377 & 0.312 & 0.476 & 42 \\
Detoxify Unbiased & insult & \textbf{0.500} & 0.378 & 0.738 & 42 \\
Perspective API & & 0.487 & 0.384 & 0.667 & 42 \\
\hline
Detoxify Original & & 0.440 & 0.314 & 0.733 & 15 \\
Detoxify Unbiased & identity attack & \textbf{0.545} & 0.414 & 0.800 & 15 \\
Perspective API & & 0.456 & 0.310 & 0.867 & 15 \\
\hline
Detoxify Original & & 0.000 & 0.000 & 0.000 & 21 \\
Detoxify Unbiased & sexually explicit & 0.392 & 0.333 & 0.476 & 21 \\
Perspective API & & \textbf{0.576} & 0.447 & 0.810 & 21 \\
\hline
\end{tabular}
\caption{Performance Metrics with Highlighted Highest F1 Scores}
\end{table}

\section{Detoxify Unbiased} \label{detoxify-unbiased}

The Unbiased Detoxify model is built on the RoBERTa-base architecture, which is a robust transformer model trained specifically for toxicity classification tasks. It was trained using the Jigsaw Unintended Bias in Toxicity Classification Challenge dataset, which includes labeled data for both general toxicity and toxicity directed toward specific identity groups. This makes it a critical tool in distinguishing between harmful general toxic language and toxic language targeting vulnerable identities, such as race, gender, and religion.

The model utilizes a combined loss function during training, which incorporates both toxicity and identity labels. This ensures that the model learns to minimize unintended bias, especially in cases where comments contain identity-related toxicity. A bias metric is also employed to evaluate how well the model performs across various identity subgroups, with the aim of reducing unfair bias in its predictions. \citep{detoxify:medium}
\chapter{Building Ray-Pipeline} \label{ray-pipeline}

The construction of the Ray pipeline for predicting toxicity in posts involved a multi-step process designed to efficiently process and analyze large-scale data. The pipeline begins by retrieving the raw data from Elasticsearch, which is the initial data source. Following this, the plaintext is extracted from the HTML code, ensuring that the textual data is clean and suitable for further processing. For efficient comparison and grouping of similar posts, a minHash is computed on the extracted plaintext. This minHash representation is used to group the posts based on Jaccard similarity, a measure of textual overlap, which helps identify and cluster posts with similar content. Now one big grouped dataset is created and will later be used to merge the toxicity predictions back to the original data.

For the toxicity prediction we build a smaller dataset by retaining only the essential columns, namely the plaintext and the group identifiers, while discarding all other unnecessary columns. Because in our study we just focus This filtered dataset is then subjected to toxicity prediction using the selected model, in this case, the Detoxify unbiased model, which assigns toxicity scores to each post across various categories.

After the toxicity predictions are completed, the results are merged back with the original, larger dataset, ensuring that the toxicity scores are aligned with the corresponding posts. Finally, the enriched dataset, now containing both the original data and the toxicity predictions, is written to a .parquet file for efficient storage and subsequent analysis. This structured pipeline not only streamlines the process of toxicity prediction but also ensures scalability and reproducibility, making it well-suited for analyzing large-scale datasets from online communities such as Mastodon.
\chapter{Discussion} \label{discussion}

The temporal analysis showed clear peaks in toxicity corresponding to major political and global events throughout 2024, particularly around the U.S. election period. The distribution of toxicity levels from highest in blocklisted instances to lowest in non-communicating ones suggests an influence of moderation practices on platform toxicity. The statistical comparisons revealed significant differences in toxicity levels between moderation practices, with particularly strong effects observed between actively moderated and poorly moderated spaces. These results collectively suggest a clear influence of moderation practices on platform toxicity.

\paragraph{Interpretation of Key Findings}
The results indicate that our toxicity prediction pipeline successfully captured patterns in the Mastodon dataset, despite limitations in model selection. The temporal peaks suggest that global events can trigger increased toxic discourse, particularly in politically charged contexts. The clear distribution of toxicity levels across moderation practices supports our hypothesis that decentralized moderation practices influence platform toxicity. Statistical evidence reveals two interesting patterns in toxicity distribution across Mastodon instances: 
\begin{enumerate}
    \item Moderated instances exhibited significantly lower toxicity levels than blocklisted instances, highlighting the impact of moderation policies.
    \item Toxicity levels were significantly higher in communicating instances compared to non-communicating ones, suggesting network-driven propagation of toxic content.
\end{enumerate}

\paragraph{Comparison with Previous Work}
Our findings align with and extend previous research on decentralized moderation. \citet{bono:2024} observed widespread blocklist usage across Mastodon instances and raised concerns about potential misuse for moderating instances that may not require intervention. Our analysis demonstrates that blocklisted instances exhibit significantly higher mean toxicity levels, while instances not communicating with blocklisted content maintain the lowest toxicity.

Our study extends the binary toxicity classification approach of \citet{al-khateeb:2022} by performing two diffrent analysis. First, rather than treating all content uniformly, we examine toxicity patterns across four distinct instance categories differentiated by their moderation policies, revealing how moderation strategies directly impact toxic discourse. Second, where previous work provided only a static snapshot, we incorporate temporal dynamics through a 12-month longitudinal analysis, capturing how toxicity fluctuates in response to real-world events. This complements event-focused studies like the study from \citet{fan:2022} on toxicity patterns during health crises on Twitter/X, suggesting that external events trigger similar toxic discourse patterns regardless of a platform's architectural design.

\paragraph{Limitations}
Our study has several limitations that should be considered when interpreting the results.

We compared three toxicity detection models using a small labeled subset of 256 toots. While the F1 scores were suboptimal for all models, we prioritized prediction confidence, ultimately selecting the Detoxify Unbiased model. This small validation set may not fully represent the diversity of toxic content in our dataset.

The analysis used a 1\% subset of approx 15~million toots rather than the full dataset due to computational and time constraints. We omitted our planned deduplication and merging methods because the subset contained fewer duplicates (50\% versus 95\% in the full dataset). The LSH-based methods for deduplication and merging require optimization for full-dataset analysis, particularly regarding memory management during LSH index queries.

Analyzing based on the classification into Blocklisted, Moderated, Communicating, and Non-communicating, defined in Section \ref{moderation:categorization}, presents a simplified perspective, as instances may be blocklisted for various reasons beyond toxicity, including spam or commercial content, rather than solely for inadequate moderation of hate speech.

In our analysis we did not pay attention to demographic data about the users. The observed effects could stem from differences in user groups (e.g., an instance dominated by far-right users versus one with left-leaning users) rather than moderation practices.

\newpage

\paragraph{Future Work}
Several promising directions emerge from this research:

First, analyzing the full dataset with optimized LSH methods would improve result reliability. The current 1\% subset, while substantial, loses information about small instances and may miss patterns visible only at full scale. Especially because the toxic instances seem to be smaller ones, as the larger ones often comitted to the Mastodon Covenant.

Second, exploring all seven toxicity labels from our model could reveal category-specific patterns. Our focus on overall toxicity provides a broad view, but deeper analysis of specific toxic behaviors (e.g., identity attacks versus threats) might yield more nuanced insights.

Third, there is an interesting pattern in the data that we could not analyze in this work. The percentage of toots marked as sensitive is very high on November~6, 2024, which is the day after the US election. This day is also the day with the highest toxicity prediction across all instances (Figure~\ref{sensitive-toots}). Since users often mark toxic content as sensitive to flag it for removal, further analysis could reveal how effective this moderation strategy is at actually reducing toxicity. 

% Bibliography
% \bibliographystyle{plainnat} % requires package natbib. An alternative is apalike
% \bibliography{literature}    % load file literature.bib
\printbibliography

\end{document}

