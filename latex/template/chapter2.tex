\chapter{Related Work}

Research on online toxicity has primarily focused on centralized social media platforms like Twitter/X and Facebook \cite{fan:2022,nicholson:2023}. These studies have examined various aspects of toxic behavior, including its diffusion patterns, impact on communities, and moderation strategies. However, DOSNs like Mastodon present unique challenges and opportunities for studying toxicity due to their federated architecture and distributed moderation systems \cite{bono:2024}. This chapter reviews relevant literature on Mastodon communities, decentralized moderation, and large-scale toxicity analysis to situate our research within the existing work.

\paragraph{Behaviour of Mastodon Communities}
Mastodon has emerged as a prominent DOSN, offering an alternative to centralized platforms by enabling users to join independent instances that form a federated network \cite{zulli:2020}. This setup allows researchers to study community behaviors, particularly how users interact or segregate across instances while this is not possible on centralised platforms \cite{zignani:2018,zulli:2020}.

The topology of Mastodon communities exhibits unique characteristics. \citet{zulli:2020} found that Mastodon instances often form around specific interests, identities, or ideologies, leading to more homogeneous communities. This clustering behavior influences information consumption patterns and user relationships, with instances developing distinct footprints based on their thematic focus \cite{la_cava:2021}. Such organizational differences suggest that toxicity patterns in Mastodon may follow different dynamics than those observed in centralized platforms.

\paragraph{Decentralized Moderation Challenges}
The federated nature of Mastodon introduces novel challenges for content moderation, as each instance maintains its own policies and enforcement mechanisms. \citet{bono:2024} found that instance administrators primarily rely on blocklisting to moderate content, preventing users from interacting with servers hosting harmful material. This decentralized approach allows for customized moderation but creates inconsistencies across the network, as blocklisting decisions vary a lot between instances \cite{nicholson:2023}. \citet{nicholson:2023} found that blocklists mainly ban instances containing spam, hate speech, or adult content. Many administrators use shared blocklists without carefully checking them first \cite{bono:2024}.

\citet{nicholson:2023} examined Mastodon's rules and discovered they focus more on preventing harassment and hate speech than similar Reddit communities. Their research showed that Mastodon's decentralized approach creates different types of rules across instances. Some instances employ rules to target specific toxic behaviour such as discrimination, while others use general community guidelines. Since each instance has its own moderation approach, users see and experience different content depending on which instance they join, making it difficult to investigate toxicity.

\paragraph{Large-Scale Toxicity Analysis}
Existing approaches to toxicity analysis have typically focused on specific events or short timeframes on centralized platforms \cite{wulczyn:2017, fan:2022,georgakopoulos:2018,badjatiya:2017}. For example, \citet{fan:2022} developed a comprehensive workflow for analyzing toxicity during health crises, combining topic modeling and network analysis to understand toxic discourse patterns. Their approach, which processed over 1.6 million tweets during the 2022 Mpox outbreak, revealed how toxicity spreads differently through mentions versus retweets and identified influential users in toxic discourse networks. As a first step toward analyzing toxicity on Mastodon, \citet{al-khateeb:2022} predicted the toxicity of 13,590~toots using the Perspective~API. The scale and distributed nature of Mastodon data require specialized processing pipelines. Existing research has highlighted the need for efficient deduplication methods when analyzing federated content, as toots often propagate across multiple instances \cite{bono:2024}.