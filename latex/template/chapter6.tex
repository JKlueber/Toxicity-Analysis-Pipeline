\chapter{Discussion} \label{discussion}

Our toxicity analysis revealed distinct patterns across Mastodon instances, aligning with the four categories we established based on moderation practices. The temporal analysis showed clear peaks in toxicity corresponding to major political and global events throughout 2024, particularly around the U.S. election period. The hierarchical distribution of toxicity levels—from highest in blocklisted instances to lowest in non-communicating ones—demonstrates a strong correlation between moderation practices and platform climate.

\paragraph{Interpretation of Results}
The results indicate that our toxicity prediction pipeline successfully captured meaningful patterns in the Mastodon dataset, despite limitations in model selection. The clear hierarchy of toxicity levels across instance categories supports our hypothesis that decentralized moderation practices significantly influence platform toxicity. The temporal peaks suggest that global events trigger increased toxic discourse, particularly in politically charged contexts. These findings validate our approach of using transformer-based models for large-scale toxicity analysis in federated social networks.

\paragraph{Comparison with Previous Work}
Our findings align with and extend previous research on decentralized moderation. \citet{bono:2024} observed widespread blocklist usage across Mastodon instances and raised concerns about potential misuse for moderating instances that may not require intervention. Our analysis demonstrates that blocklisted instances exhibit significantly higher mean toxicity levels, while instances not communicating with blocklisted content maintain the lowest toxicity. This evidence suggests that blocklists effectively moderate genuinely harmful instances when properly implemented. 

The temporal dimension of our analysis provides new insights into toxicity evolution during real-world events, complementing event-focused studies like \citet{fan:2022} on toxicity patterns during health crises.

\paragraph{Limitations}
Our study has several limitations that should be considered when interpreting the results.

\subparagraph{Model Selection}
We compared three toxicity detection models using a small labeled subset of 256 toots. While the F1 scores were suboptimal for all models, we prioritized prediction confidence, ultimately selecting the Detoxify Unbiased model. This small validation set may not fully represent the diversity of toxic content in our dataset. However, the model's performance in identifying toxicity trends across instances suggests it was suitable for our analysis goals.

\subparagraph{Data Scope}
The analysis used a 1\% subset (approximately 18 million toots) rather than the full dataset due to computational constraints. We omitted our planned deduplication and merging methods because the subset contained fewer duplicates (50\% versus 95\% in the full dataset). The LSH-based methods for deduplication and merging require optimization for full-dataset analysis, particularly regarding memory management during index queries.

\paragraph{Future Work}
Several promising directions emerge from this research:

First, analyzing the full dataset with optimized LSH methods would improve result reliability. The current 1\% subset, while substantial, loses information about small instances and may miss patterns visible only at full scale. Especially because the toxic instances are the smaller ones.

Second, exploring all seven toxicity labels from our model could reveal category-specific patterns. Our focus on overall toxicity provides a broad view, but deeper analysis of specific toxic behaviors (e.g., identity attacks versus threats) might yield more nuanced insights.

Third, there is an interesting pattern in the data that we could not analyze in this work. The number of toots marked as sensitive is very high on November~6, 2024, which is the day after the US election. This day is also the day with the highest toxicity prediction across all instances (Figure~\ref{sensitive-toots}). Sensitive labels are typically used by moderators to flag toxic content for removal. However, some sexually-oriented instances apply this label mainly for sexual content. When we focus on those with a significant increase in sensitive-labeled toots on November~6, 2024, the day of peak toxicity and sensitive-labeled toots, this approach helps distinguish moderation-related labeling from other uses. This could be done in future work to analyze this kind of moderation.



