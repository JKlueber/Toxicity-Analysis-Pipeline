\chapter{Mastodon Dataset} \label{mastodon-dataset}
The dataset used in this study consists of public Mastodon posts (called ``toots'') collected from a federated network of instances. The data was gathered using a distributed crawler developed by \citet{ernst:2024}, which collects posts from Mastodon instances' public timelines via their REST API while respecting user privacy settings. The crawler stores the collected data in an Elasticsearch cluster with careful attention to ethical considerations, never publishing raw user data.

As reported by \citet{ernst:2024} in November~2024, the corpus contained 3.6~billion posts collected over 301~days from 1,081~instances. The complete dataset includes detailed metadata about each post and relationships between instances, with key fields described in Table~\ref{dataset-fields}. After deduplication, this resulted in 174~million unique posts, indicating a duplication rate of approximately 95\%\@, which reflects the federated nature of Mastodon where content is shared across instances. While the crawler has continued running since that report, our analysis focuses specifically on English-language toots from the complete year~2024, collected from 1,000~fully crawled instances.

\begin{table}[tb]
    \centering\small
    \renewcommand{\arraystretch}{1.3}
    \begin{tabularx}{\textwidth}{lX}
        \toprule
        \textbf{Field} & \textbf{Description} \\
        \midrule
        \texttt{id} & Unique identifier for each toot \\
        \texttt{content} & The post content in HTML format (converted to plain text for analysis) \\
        \texttt{crawled\_from\_instance} & The instance where the toot was observed \\
        \texttt{instance} & The home instance of the posting user \\
        \texttt{is\_local} & Boolean indicating whether the post originated on the crawled instance \\
        \texttt{created\_at} & Timestamp of post creation \\
        \texttt{sensitive} & Flag marking potentially sensitive content \\
        \texttt{spoiler\_text} & Content warnings or spoiler alerts \\
        \bottomrule
    \end{tabularx}
    \caption{Key fields available in the Mastodon dataset with their descriptions.}
    \label{dataset-fields}
\end{table}

The original corpus analysis revealed significant diversity in the data sources. While most posts originate from Mastodon instances, notable contributions come from other federated platforms like Misskey and RSS~Parrot. English (34.71\%) and Japanese (23.39\%) dominate the language distribution, with all other languages below 5\% each. Approximately 18\% of posts contain media attachments (mostly images), while 24\% include URL preview cards. Because our toxicity detection models just analyze text content, we removed those with media attachments. As well, we filtered out reblogs to reduce the dataset and focus on original posts. The final dataset contains around 1.8~billion toots. Becuase of the lack of time, we only analyze a 10\% subsample of the dataset in this study.